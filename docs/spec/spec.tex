\documentclass[oneside]{memoir}

\usepackage{enumitem}
\usepackage[T1]{fontenc}
\usepackage[margin=3cm]{geometry}
\usepackage[hidelinks, linktoc=all]{hyperref}
\usepackage{microtype}
\usepackage{multicol}
\usepackage{textcomp}
\usepackage{underscore}

\newcommand{\code}{\texttt}
\newcommand{\nonterminal}{\textit}
\renewcommand{\rule}[1]{\nonterminal{#1} \textrightarrow{}}
\newcommand{\optional}{\textsubscript{opt}}
\newenvironment{grammar}{\begin{quote}}{\end{quote}}

\DisableLigatures{encoding = T1, family = tt*}
\nonzeroparskip
\setlength{\parindent}{0pt}
\setcounter{tocdepth}{3}
\setcounter{secnumdepth}{3}

\begin{document}

\title{Delta Language Specification}
\author{Emil Laine \\ \href{mailto:emil.laine@helsinki.fi}{emil.laine@helsinki.fi}}
\maketitle

This document describes the syntax and semantics of the Delta programming language.

\textbf{Note:} This document is incomplete and parts of it may be out of date.

\tableofcontents

%!TEX root = spec.tex

\chapter{Lexical structure}

\section{Keywords}

The following keywords are reserved and can't be used as identifiers.

\begin{multicols}{3}
\begin{itemize}[label={}]
\item \code{addressof}
\item \code{break}
\item \code{case}
\item \code{catch}
\item \code{const}
\item \code{continue}
\item \code{default}
\item \code{defer}
\item \code{deinit}
\item \code{do}
\item \code{else}
\item \code{enum}
\item \code{extern}
\item \code{fallthrough}
\item \code{false}
\item \code{for}
\item \code{goto}
\item \code{if}
\item \code{import}
\item \code{in}
\item \code{init}
\item \code{inline}
\item \code{interface}
\item \code{mutable}
\item \code{mutating}
\item \code{null}
\item \code{private}
\item \code{public}
\item \code{return}
\item \code{sizeof}
\item \code{static}
\item \code{struct}
\item \code{switch}
\item \code{this}
\item \code{throw}
\item \code{throws}
\item \code{true}
\item \code{try}
\item \code{typealias}
\item \code{undefined}
\item \code{var}
\item \code{while}
\item \code{_}
\end{itemize}
\end{multicols}

\section{Operators and delimiters}

Binary arithmetic operators:

\begin{multicols}{4}
\begin{itemize}[label={}]
\item \code{+}
\item \code{-}
\item \code{*}
\item \code{/}
\item \code{**}
\item \code{\%}
\item \code{\&}
\item \code{\&\&}
\item \code{|}
\item \code{||}
\item \code{\char`\^}
\item \code{<<}
\item \code{>>}
\item \code{+=}
\item \code{-=}
\item \code{*=}
\item \code{/=}
\item \code{**=}
\item \code{\%=}
\item \code{\&=}
\item \code{\&\&=}
\item \code{|=}
\item \code{||=}
\item \code{\char`\^=}
\item \code{<<=}
\item \code{>>=}
\end{itemize}
\end{multicols}

Binary comparison operators:

\begin{multicols}{4}
\begin{itemize}[label={}]
\item \code{==}
\item \code{!=}
\item \code{<}
\item \code{>}
\item \code{<=}
\item \code{>=}
\end{itemize}
\end{multicols}

Miscellaneous binary operators:

\begin{multicols}{4}
\begin{itemize}[label={}]
\item \code{=}
\item \code{..}
\item \code{...}
\end{itemize}
\end{multicols}

Unary prefix operators:

\begin{multicols}{4}
\begin{itemize}[label={}]
\item \code{+}
\item \code{-}
\item \code{*}
\item \code{\&}
\item \code{!}
\item \code{\~}
\end{itemize}
\end{multicols}

Unary postfix operators:

\begin{multicols}{4}
\begin{itemize}[label={}]
\item \code{++}
\item \code{--}
\item \code{!}
\end{itemize}
\end{multicols}

Delimiters:

\begin{multicols}{4}
\begin{itemize}[label={}]
\item \code{(}
\item \code{[}
\item \code{\{}
\item \code{)}
\item \code{]}
\item \code{\}}
\item \code{.}
\item \code{,}
\item \code{:}
\item \code{;}
\item \code{->}
\end{itemize}
\end{multicols}

From the above sets of operators, the following are overloadable by user code:
\code{+} (both unary and binary), \code{-} (both unary and binary), \code{*},
\code{/}, \code{\%}, \code{==}, \code{<}.

\section{Comments}

Delta has two kinds of comments:

\begin{itemize}
\item Line comments that start with \code{//} and continue until the end of the line.
\item Block comments that start with \code{/*} and end with \code{*/}. Block comments can be nested.
\end{itemize}

\section{Literals}

\subsection{Integer literal}

\begin{grammar}
\rule{binary-digit} \code{0} | \code{1}\\
\rule{octal-digit} \code{0} | \code{1} | \code{2} | \code{3} | \code{4} | \code{5} | \code{6} | \code{7}\\
\rule{decimal-digit} \code{0} | \code{1} | \code{2} | \code{3} | \code{4} | \code{5} | \code{6} | \code{7} | \code{8} | \code{9}\\
\rule{nonzero-decimal-digit} \code{1} | \code{2} | \code{3} | \code{4} | \code{5} | \code{6} | \code{7} | \code{8} | \code{9}\\
\rule{lowercase-hex-digit} \code{0} | \code{1} | \code{2} | \code{3} | \code{4} | \code{5} | \code{6} | \code{7} | \code{8} | \code{9} | \code{a} | \code{b} | \code{c} | \code{d} | \code{e} | \code{f}\\
\rule{uppercase-hex-digit} \code{0} | \code{1} | \code{2} | \code{3} | \code{4} | \code{5} | \code{6} | \code{7} | \code{8} | \code{9} | \code{A} | \code{B} | \code{C} | \code{D} | \code{E} | \code{F}

\rule{binary-integer-literal} \code{0b}\nonterminal{binary-digit}+\\
\rule{octal-integer-literal} \code{0o}\nonterminal{octal-digit}+\\
\rule{decimal-integer-literal} \nonterminal{nonzero-decimal-digit} \nonterminal{decimal-digit}* | \code{0}\\
\rule{hex-integer-literal} \code{0x}(\nonterminal{lowercase-hex-digit}+ | \nonterminal{uppercase-hex-digit}+)

\rule{integer-literal} \nonterminal{binary-integer-literal}\\
\rule{integer-literal} \nonterminal{octal-integer-literal}\\
\rule{integer-literal} \nonterminal{decimal-integer-literal}\\
\rule{integer-literal} \nonterminal{hex-integer-literal}
\end{grammar}

\subsection{Floating-point literal}

Floating-point literals have the following form:

\begin{grammar}
\rule{floating-point-literal} \nonterminal{nonzero-decimal-digit} \nonterminal{decimal-digit}*\code{.}\nonterminal{decimal-digit}+\\
\rule{floating-point-literal} \code{0}\code{.}\nonterminal{decimal-digit}+
\end{grammar}

\subsection{Boolean literal}

\begin{grammar}
\rule{boolean-literal} \code{true} | \code{false}
\end{grammar}

\subsection{Null literal}

\begin{grammar}
\rule{null-literal} \code{null}
\end{grammar}

\subsection{String literal}

\begin{grammar}
\rule{string-literal} \code{"}(\nonterminal{character} | \nonterminal{interpolated-expression})*\code{"}\\
\rule{interpolated-expression} \code{\$\{} \nonterminal{expression} \code{\}}
\end{grammar}

\subsection{Array literal}

\begin{grammar}
\rule{array-literal} \code{[} \nonterminal{elements} \code{]}
\end{grammar}

where \nonterminal{elements} is a comma-separated list of zero or more expressions of the
same type.

\subsection{Tuple literal}

\begin{grammar}
\rule{tuple-literal} \code{(} \nonterminal{elements} \code{)}
\end{grammar}

where \nonterminal{elements} is a comma-separated list of zero or more
\nonterminal{tuple-literal-elements}:

\begin{grammar}
\rule{tuple-literal-element} \nonterminal{identifier} \code{:} \nonterminal{expression}\\
\rule{tuple-literal-element} \nonterminal{identifier}
\end{grammar}

The second form is a shorthand for \nonterminal{tuple-literal-elements} of the
form `\nonterminal{identifier} \code{:} \nonterminal{identifier}'.

\section{Identifiers}

\begin{grammar}
\rule{identifier-first-character} upper- or lowercase letter \code{A} through \code{Z}\\
\rule{identifier-first-character} \code{_}

\rule{identifier-character} \nonterminal{identifier-first-character}\\
\rule{identifier-character} \code{0} | \code{1} | \code{2} | \code{3} | \code{4} | \code{5} | \code{6} | \code{7} | \code{8} | \code{9}

\rule{identifier} \nonterminal{identifier-first-character} \nonterminal{identifier-character}*
\end{grammar}

%!TEX root = spec.tex

\chapter{Types}

\begin{grammar}
\rule{type} \nonterminal{basic-type}\\
\rule{type} \nonterminal{pointer-type}\\
\rule{type} \nonterminal{optional-type}\\
\rule{type} \nonterminal{array-type}\\
\rule{type} \nonterminal{function-type}\\
\rule{type} \nonterminal{tuple-type}\\
\rule{type} \code{mutable} \nonterminal{basic-type}\\
\rule{type} \code{mutable} \nonterminal{function-type}\\
\rule{type} \code{mutable} \nonterminal{tuple-type}
\end{grammar}

\section{Basic types}

\begin{grammar}
\rule{basic-type} \nonterminal{identifier}\\
\rule{basic-type} \nonterminal{identifier} \code{<} \nonterminal{generic-argument-list} \code{>}\\
\rule{generic-argument-list} comma-separated list of one or more \nonterminal{types}
\end{grammar}

\subsection{Integer types}

There are eight fixed-width integer types: \code{int8}, \code{int16},
\code{int32}, \code{int64}, and their unsigned counterparts \code{uint8},
\code{uint16}, \code{uint32}, \code{uint64}. The language also provides:

\begin{itemize}
\item \code{byte} and \code{ubyte}, which have at least 8 bits
\item \code{short} and \code{ushort}, which have at least 16 bits
\item \code{int} and \code{uint}, which have at least 32 bits
\item \code{long} and \code{ulong}, which have at least 64 bits
\end{itemize}

Overflow is undefined for all integer types, both signed and unsigned, to aid
optimization. Overflow checks are enabled by default, and can be disabled with a
compiler option, or by compiling in unchecked mode. The standard library
provides arithmetic functions that have defined behavior on overflow.

\subsection{Floating-point types}

There are three fixed-width floating-point types: \code{float32},
\code{float64}, and \code{float80}. The language also provides:

\begin{itemize}
\item \code{float}, which has at least 32 bits
\item \code{double}, which has at least 64 bits
\end{itemize}

\subsection{Struct types}

Structs are composite data types which can be defined using the \code{struct}
keyword.

\subsection{Interface types}

The \code{interface} keyword declares an interface, i.e. a set of requirements
(member functions and properties). Types that are declared to implement an
interface \code{I} and fulfill \code{I}'s requirements can be used as values for
a variable of type \code{I}. This enables runtime polymorphism. Like structs,
interfaces may be generic.

\section{Pointer types}

Pointers are values that point to other values. They can be reassigned to point
to another value (if the pointer type itself is declared as \code{mutable}), but
they must always refer to some value, i.e. they cannot be null by default
(nullable pointers can be created using the optional type, see below). Member
access, member function calls, and subscript operations via pointers are
allowed: they will be forwarded to the pointee value.

\begin{grammar}
\rule{pointer-type} \nonterminal{pointee-type} \code{*}\\
\rule{pointer-type} \nonterminal{pointee-type} \code{mutable} \code{*}\\
\rule{pointee-type} \nonterminal{type}
\end{grammar}

The \nonterminal{pointee-type} may be mutable. Prefixing the \code{*} with
\code{mutable} makes the \nonterminal{pointer-type} itself mutable.

\begin{samepage}
Pointer arithmetic is supported in the form of the following operations:

\begin{itemize}
\item \nonterminal{pointer} \code{+} \nonterminal{integer}
\item \nonterminal{pointer} \code{+=} \nonterminal{integer}
\item \nonterminal{pointer} \code{++}
\item \nonterminal{pointer} \code{-} \nonterminal{integer}
\item \nonterminal{pointer} \code{-=} \nonterminal{integer}
\item \nonterminal{pointer} \code{--}
\item \nonterminal{pointer} \code{-} \nonterminal{pointer}
\end{itemize}
\end{samepage}

\section{Array types}

\begin{grammar}
\rule{array-type-with-constant-size} \nonterminal{element-type} \code{[} \nonterminal{size} \code{]}\\
\rule{array-type-with-runtime-size} \nonterminal{element-type} \code{[} \code{]}\\
\rule{array-type-with-unknown-size} \nonterminal{element-type} \code{[} \code{?} \code{]}\\
\rule{element-type} \nonterminal{type}
\end{grammar}

\nonterminal{array-type-with-constant-size} represents a contiguous block of
\nonterminal{size} elements of type \nonterminal{element-type}.
\nonterminal{array-type-with-runtime-size} is conceptually a pointer-and-size
pair. \nonterminal{array-type-with-unknown-size} can only be used through a
pointer; such pointers are memory-layout-compatible with pointers to
\nonterminal{element-type}, primarily for C interoperability.
\nonterminal{array-type-with-unknown-size} is the only array type for which
index operations are not guaranteed to be bounds-checked.

The \nonterminal{element-type} may be mutable.

\section{Optional type}

\begin{quote}
``I call it my billion-dollar mistake. It was the invention of the null
reference in 1965. At that time, I was designing the first comprehensive type
system for references in an object oriented language (ALGOL W). My goal was to
ensure that all use of references should be absolutely safe, with checking
performed automatically by the compiler. But I couldn't resist the temptation to
put in a null reference, simply because it was so easy to implement. This has
led to innumerable errors, vulnerabilities, and system crashes, which have
probably caused a billion dollars of pain and damage in the last forty years.''

--- C. A. R. Hoare
\end{quote}

An object of the optional type \code{T?} (where \code{T} is an arbitrary type)
may contain a value of type \code{T} or the value \code{null}.

\begin{grammar}
\rule{optional-type} \nonterminal{wrapped-type} \code{?}\\
\rule{wrapped-type} \nonterminal{type}
\end{grammar}

\section{Function types}

Function types are written out as follows:

\begin{grammar}
\rule{function-type} \code{(} \nonterminal{parameter-type-list} \code{)} \code{->} \nonterminal{return-type}\\
\rule{parameter-type-list} comma-separated list of zero or more \nonterminal{types}\\
\rule{return-type} \nonterminal{type}
\end{grammar}

The types in the \nonterminal{parameter-type-list} and \nonterminal{return-type}
may not have a top-level mutable modifier.

\section{Tuple types}

\begin{grammar}
\rule{tuple-type} \code{(} \nonterminal{tuple-element-list} \code{)}\\
\rule{tuple-element-list} comma-separated list of one or more \nonterminal{tuple-elements}\\
\rule{tuple-element} \nonterminal{name} \code{:} \nonterminal{type}
\end{grammar}

Tuples behave like structs, but they're defined inline. Tuples are intended as
a lightweight alternative for situations where defining a whole new struct feels
overkill or inappropriate, e.g. returning multiple values from a function.

While struct types are considered the same only if they have the same name,
tuple types are considered the same if their structure is the same, i.e. if they
have the same number of elements in the same order with the same names and
types.

\subsection{Tuple unpacking}

The elements of a tuple value may be unpacked into individual variables as
follows:

\begin{grammar}
\rule{tuple-unpack-statement} \nonterminal{variable-list} \code{=} \nonterminal{tuple-expression} \code{;}
\end{grammar}

\nonterminal{variable-list} is a comma-separated list of one or more variable
names. The variable names must match the element names of the
\nonterminal{tuple-expression}, and be in the same order.

%!TEX root = spec.tex

\chapter{Declarations}

\section{Variables}

Variable declarations introduce a new variable into the enclosing scope. The
syntax is as follows:

\begin{grammar}
\rule{implicitly-typed-variable-definition} \code{var} \nonterminal{variable-name} \code{=} \nonterminal{initializer} \code{;}\\
\rule{explicitly-typed-variable-definition} \code{var} \nonterminal{variable-name} \code{:} \nonterminal{type} \code{=} \nonterminal{initializer} \code{;}\\
\rule{variable-declaration} \code{var} \nonterminal{variable-name} \code{:} \nonterminal{type} \code{;}
\end{grammar}

If \nonterminal{type} is present, the variable has the specified type. The
compiler ensures that the given \nonterminal{initializer} is compatible with
this type. If no \nonterminal{type} is given, the compiler will infer the type
of the variable from the \nonterminal{initializer}. The
\nonterminal{initializer} is an expression that provides the initial value for
the variable.

If \nonterminal{type} has been specified, \nonterminal{initializer} may also be
the keyword \code{undefined}, in which case the variable is not initialized and
all use-before-initialization warnings for the variable will be suppressed.
Reading from an uninitialized variable causes undefined behavior.

In \nonterminal{variable-declaration}, the variable is declared but not
initialized. This allows delayed initialization, which causes the compiler to
enforce that the variable is always initialized properly before its value is
accessed.

\section{Functions}

A function can be defined with either of the following syntaxes:

\begin{grammar}
\code{def} \nonterminal{function-name} \code{(} \nonterminal{parameter-list} \code{)} \code{\{} \nonterminal{function-body} \code{\}}\\
\code{def} \nonterminal{function-name} \code{(} \nonterminal{parameter-list} \code{)} \code{:} \nonterminal{return-type} \code{\{} \nonterminal{function-body} \code{\}}
\end{grammar}

The return type of the first version is \code{void}. The
\nonterminal{parameter-list} is a comma-separated list of
\nonterminal{parameters}:

\begin{grammar}
\rule{parameter} \nonterminal{parameter-name} \code{:} \nonterminal{parameter-type}
\end{grammar}

\nonterminal{parameter-name} is an identifier specifying the name of the
parameter. A function cannot have multiple parameters with the same name.

\nonterminal{return-type} defines what kind of values the function can return.
This may be a tuple type to allow the function to return multiple values without
having to define a whole new struct type.

Parameter and return types may not have a top-level \code{mutable} modifier.

A function declaration may optionally be prefixed with any number of
\hyperref[sec:function-specifiers]{\nonterminal{function-specifiers}}.
% TODO: Add full section numbers to section links for printing.

\subsection{Member functions}

Member functions are just like normal functions, except that they receive an
additional parameter, called the "receiver", on the left-hand-side of the
function call, separated by a period:

\begin{grammar}
\rule{member-function-call} \nonterminal{receiver} \code{.} \nonterminal{member-function-name} \code{(} \nonterminal{argument-list} \code{)}
\end{grammar}

Member functions are defined with the same syntax as non-member functions, but
are written inside a type declaration. That type declaration defines the member
function's receiver type. Inside member functions, the receiver can be accessed
with the keyword \code{this}.

\subsubsection{Initializers}

Initializers are a special kind of member functions that are used for
initializing newly created objects.

\begin{grammar}
\rule{initializer-definition} \code{init} \code{(} \nonterminal{parameter-list} \code{)} \code{\{} \nonterminal{body} \code{\}}
\end{grammar}

Initializers can be invoked with the following syntax:

\begin{grammar}
\rule{initializer-call} \nonterminal{receiver-type} \code{(} \nonterminal{argument-list} \code{)}
\end{grammar}

The \nonterminal{initializer-call} expression returns a new instance of the
specified type that has been initialized by calling the initializer function
with a matching parameter list.

\subsubsection{Deinitializers}

Deinitializers are another special kind of member functions. They are
automatically called on objects when they're destroyed, but can also be called
explicitly. They can be used e.g. to deallocate resources allocated in an
initializer. They are declared as follows:

\begin{grammar}
\rule{deinitializer-definition} \code{deinit} \code{(} \code{)} \code{\{} \nonterminal{body} \code{\}}
\end{grammar}

\subsection{Private and public functions}

Both member functions and global functions may be declared private or public by
prefixing the function definition with the keyword \code{private} or
\code{public}. Private functions are only accessible from the file they're
declared in. Public functions are accessible from anywhere, including other
modules. Functions not marked private or public are
\nonterminal{module-private}, i.e. only accessible within the module they're
declared in.

\subsection{Function specifiers}
\label{sec:function-specifiers}

\begin{grammar}
\rule{function-specifier} \code{inline}
\end{grammar}

\subsubsection{\code{inline}}

A function defined with the \code{inline} keyword is an \textit{inline
function}. Inline functions are guaranteed to be inlined when compiling in debug
mode (without optimizations). When compiling with optimizations, the compiler
may choose to not inline an inline function.

\section{Structs}

Structs are defined as follows:

\begin{grammar}
\rule{struct-definition} \code{struct} \nonterminal{struct-name} \code{\{} \nonterminal{member-list} \code{\}}
\end{grammar}

\nonterminal{struct-name} becomes the name of the struct.
\nonterminal{member-list} is a sequence of
\nonterminal{member-variable-declarations} and
\nonterminal{member-function-declarations}. Structs can be declared to implement
interfaces by listing the interfaces after a \code{:} following the struct name:

\begin{grammar}
\code{struct} \nonterminal{struct-name} \code{:} \nonterminal{interface-list} \code{\{} \nonterminal{member-list} \code{\}}
\end{grammar}

The \nonterminal{interface-list} is a comma-separated list of one or more
interface names. The compiler will emit an error if the struct doesn't fulfill
all the requirements of a specified interface.

\subsection{Member variables}

Structs can contain member variables. The syntax of a member variable definition
is as follows:

\begin{grammar}
\rule{member-variable-declaration} \code{var} \nonterminal{member-variable-name} \code{:} \nonterminal{type} \code{;}
\end{grammar}

\subsection{Generic structs}

Generic structs can be declared as follows:

\begin{grammar}
\code{struct} \nonterminal{struct-name} \code{<} \nonterminal{generic-parameter-list} \code{>} \code{\{} \nonterminal{member-list} \code{\}}
\end{grammar}

where \nonterminal{generic-parameter-list} is a comma separated list of one or
more \nonterminal{generic-parameters}:

\begin{grammar}
\rule{generic-parameter} \nonterminal{generic-type-parameter}\\
\rule{generic-type-parameter} \nonterminal{identifier}
\end{grammar}

The identifier of a \nonterminal{generic-type-parameter} serves as a placeholder
for types used to instantiate the generic struct.

%!TEX root = spec.tex

\chapter{Statements}

\section{Assignment statement}

\begin{grammar}
\rule{assignment-statement} \nonterminal{lvalue-expression} \code{=} \nonterminal{expression} \code{;}\\
\rule{assignment-statement} \code{_} \code{=} \nonterminal{expression} \code{;}
\end{grammar}

Assignments in Delta don't return any value. This applies to compound
assignments as well, including \code{++} and \code{--} (see below). Furthermore,
this obsoletes the two different forms of \code{++} and \code{--}, so only the
postfix versions are valid as syntactic sugar for \code{+= 1} and \code{-= 1},
respectively.

The assignment to \code{_}, called the \nonterminal{discarding-assignment}, can
be used to ignore the result of the right-hand side expression, suppressing any
compilation errors or warnings that would otherwise be emitted.

\section{Increment and decrement statements}

\begin{grammar}
\rule{increment-statement} \nonterminal{lvalue-expression} \code{++} \code{;}\\
\rule{decrement-statement} \nonterminal{lvalue-expression} \code{--} \code{;}
\end{grammar}

\section{Block}

\begin{grammar}
\rule{block} \code{\{} \nonterminal{statement}* \code{\}}
\end{grammar}

\section{\code{if} statement}

\begin{grammar}
\rule{if-statement} \code{if} \code{(} \nonterminal{expression} \code{)} \nonterminal{block} (\code{else} \nonterminal{block})\optional\\
\rule{if-statement} \code{if} \code{(} \nonterminal{expression} \code{)} \nonterminal{block} \code{else} \nonterminal{if-statement}
\end{grammar}

\section{\code{return} statement}

\begin{grammar}
\rule{return-statement} \code{return} \nonterminal{expression} \code{;}
\end{grammar}

\section{\code{for} statement}

The \code{for} statement loops over a range. The syntax is as follows:

\begin{grammar}
\rule{for-statement} \code{for} \code{(} \nonterminal{let-or-var} \nonterminal{identifier} \code{in} \nonterminal{range-expression} \code{)} \nonterminal{block}
\end{grammar}

\section{\code{while} statement}

The \code{while} statement loops until a condition evaluates to \code{false}.
The syntax is as follows:

\begin{grammar}
\rule{while-statement} \code{while} \code{(} \nonterminal{condition} \code{)} \nonterminal{block}
\end{grammar}

\section{\code{switch} statement}

\begin{grammar}
\rule{switch-statement} \code{switch} \code{(} \nonterminal{expression} \code{)} \code{\{} \nonterminal{case}+ \code{\}}\\
\rule{case} \code{case} \nonterminal{expression} \code{:} \nonterminal{statement}+\\
\rule{case} \code{default} \code{:} \nonterminal{statement}+
\end{grammar}

In addition to integer types, the \code{switch} statement can be used to match
strings.

The cases in a \code{switch} statement don't fall through by default. The
fall-through behavior can be enabled for a individual cases with the
\code{fallthrough} keyword.

\code{switch} statements must be exhaustive if \nonterminal{expression} is of an
enum type. This is enforced by the compiler.

\section{\code{defer} statement}

The \code{defer} statement has the following syntax:

\begin{grammar}
\rule{defer-statement} \code{defer} \nonterminal{block}
\end{grammar}

The \nonterminal{block} will be executed when leaving the scope where the
\nonterminal{defer-statement} is located. Multiple deferred blocks are executed
in the reverse of the order they were declared in. Return statements are
disallowed inside the defer block.

%!TEX root = spec.tex

\chapter{Expressions}

\section{Unary expressions}

\begin{grammar}
\rule{prefix-unary-expression} \nonterminal{operator} \nonterminal{operand}\\
\rule{postfix-unary-expression} \nonterminal{operand} \nonterminal{operator}
\end{grammar}

\subsection{Unwrap expression}

\begin{grammar}
\rule{unwrap-expression} \nonterminal{operand} \code{!}
\end{grammar}

The \nonterminal{unwrap-expression} takes an operand of an optional type, and
returns the value wrapped by the optional. If the operand is null, an assertion
error will be triggered, except in unchecked mode, where the compiler may assume
that the operand is never null.

\section{Binary expression}

\begin{grammar}
\rule{binary-expression} \nonterminal{left-hand-side} \nonterminal{binary-operator} \nonterminal{right-hand-side}
\end{grammar}

\section{Conditional expression}

\begin{grammar}
\rule{conditional-expression} \nonterminal{condition} \code{?} \nonterminal{then-expression} \code{:} \nonterminal{else-expression}
\end{grammar}

\section{Member access expression}

\begin{grammar}
\rule{member-access-expression} \nonterminal{expression} \code{.} \nonterminal{identifier}
\end{grammar}

\section{Subscript expression}

\begin{grammar}
\rule{subscript-expression} \nonterminal{expression} \code{[} \nonterminal{expression} \code{]}
\end{grammar}

\section{Function call expression}

\begin{grammar}
\rule{call-expression} \nonterminal{expression} \code{(} \nonterminal{argument-list} \code{)}
\end{grammar}

\nonterminal{argument-list} is a comma-separated list of zero or more
\nonterminal{argument-specifiers}:

\begin{grammar}
\rule{argument-specifier} \nonterminal{unnamed-argument} | \nonterminal{named-argument}\\
\rule{unnamed-argument} \nonterminal{expression}\\
\rule{named-argument} \nonterminal{argument-name} \code{:} \nonterminal{expression}
\end{grammar}

\nonterminal{argument-name} is an identifier specifying the name of the
parameter the argument \nonterminal{expression} is being assigned to.

\section{Range expression}

\begin{grammar}
\rule{exclusive-range-expression} \nonterminal{lower-bound} \code{..} \nonterminal{upper-bound}\\
\rule{inclusive-range-expression} \nonterminal{lower-bound} \code{...} \nonterminal{upper-bound}
\end{grammar}

\section{Closure expression}

\begin{grammar}
\rule{closure-expression} \code{(} \nonterminal{parameter-list} \code{)} \code{->} \nonterminal{expression}\\
\rule{closure-expression} \code{(} \nonterminal{parameter-list} \code{)} \code{->} \nonterminal{block}
\end{grammar}

Specifying the type for parameters in a closure \nonterminal{parameter-list} is
optional. Omitting the type (and the corresponding colon) causes the type for
that parameter to be inferred from the context.

If the closure \nonterminal{parameter-list} only contains one parameter, the
enclosing parentheses may be omitted.

%!TEX root = spec.tex

\chapter{Standard library}

\section{Types}

\subsection{String types}

The type \code{String} holds sequences of Unicode characters.

\subsection{Range types}

The standard library defines the following two generic types to represent
ranges:

\begin{itemize}
\item \code{Range<T>} for ranges with an exclusive upper bound
\item \code{ClosedRange<T>} for ranges with an inclusive upper bound
\end{itemize}


\end{document}
