%!TEX root = spec.tex

\chapter{Declarations}

\section{Variables}

Variable declarations introduce a new variable into the enclosing scope. The
syntax is as follows:

\begin{grammar}
\rule{implicitly-typed-variable-definition} \nonterminal{let-or-var} \nonterminal{variable-name} \code{=} \nonterminal{initializer} \code{;}\\
\rule{explicitly-typed-variable-definition} \nonterminal{let-or-var} \nonterminal{variable-name} \code{:} \nonterminal{type} \code{=} \nonterminal{initializer} \code{;}\\
\rule{variable-declaration} \nonterminal{let-or-var} \nonterminal{variable-name} \code{:} \nonterminal{type} \code{;}
\end{grammar}

If \nonterminal{type} is present, the variable has the specified type. The
compiler ensures that the given \nonterminal{initializer} is compatible with
this type. If no \nonterminal{type} is given, the compiler will infer the type
of the variable from the \nonterminal{initializer}. The
\nonterminal{initializer} is an expression that provides the initial value for
the variable.

If \nonterminal{type} has been specified, \nonterminal{initializer} may also be
the keyword \code{undefined}, in which case the variable is not initialized and
all use-before-initialization warnings for the variable will be suppressed.
Reading from an uninitialized variable causes undefined behavior.

In \nonterminal{variable-declaration}, the variable is declared but not
initialized. This allows delayed initialization, which causes the compiler to
enforce that the variable is always initialized properly before its value is
accessed.

\section{Functions}

A function can be defined with either of the following syntaxes:

\begin{grammar}
\code{func} \nonterminal{function-name} \code{(} \nonterminal{parameter-list} \code{)} \code{\{} \nonterminal{function-body} \code{\}}\\
\code{func} \nonterminal{function-name} \code{(} \nonterminal{parameter-list} \code{)} \code{->} \nonterminal{return-type} \code{\{} \nonterminal{function-body} \code{\}}
\end{grammar}

The return type of the first version is \code{void}. The
\nonterminal{parameter-list} is a comma-separated list of
\nonterminal{parameters}:

\begin{grammar}
\rule{parameter} \nonterminal{parameter-name} \code{:} \nonterminal{parameter-type}
\end{grammar}

\nonterminal{parameter-name} is an identifier specifying the name of the
parameter. A function cannot have multiple parameters with the same name.

\nonterminal{return-type} defines what kind of values the function can return.
This may be a tuple type to allow the function to return multiple values without
having to define a whole new struct type.

Parameter and return types may not have a top-level \code{mutable} modifier.

A function declaration may optionally be prefixed with any number of
\hyperref[sec:function-specifiers]{\nonterminal{function-specifiers}}.
% TODO: Add full section numbers to section links for printing.

\subsection{Member functions}

Member functions are just like normal functions, except that they receive an
additional parameter, called the "receiver", on the left-hand-side of the
function call, separated by a period:

\begin{grammar}
\rule{member-function-call} \nonterminal{receiver} \code{.} \nonterminal{member-function-name} \code{(} \nonterminal{argument-list} \code{)}
\end{grammar}

Member functions are defined with the same syntax as non-member functions, but
are written inside a type declaration. That type declaration defines the member
function's receiver type. Inside member functions, the receiver can be accessed
with the keyword \code{this}.

\subsubsection{Initializers}

Initializers are a special kind of member functions that are used for
initializing newly created objects.

\begin{grammar}
\rule{initializer-definition} \code{init} \code{(} \nonterminal{parameter-list} \code{)} \code{\{} \nonterminal{body} \code{\}}
\end{grammar}

Initializers can be invoked with the following syntax:

\begin{grammar}
\rule{initializer-call} \nonterminal{receiver-type} \code{(} \nonterminal{argument-list} \code{)}
\end{grammar}

The \nonterminal{initializer-call} expression returns a new instance of the
specified type that has been initialized by calling the initializer function
with a matching parameter list.

\subsubsection{Deinitializers}

Deinitializers are another special kind of member functions. They are
automatically called on objects when they're destroyed, but can also be called
explicitly. They can be used e.g. to deallocate resources allocated in an
initializer. They are declared as follows:

\begin{grammar}
\rule{deinitializer-definition} \code{deinit} \code{(} \code{)} \code{\{} \nonterminal{body} \code{\}}
\end{grammar}

\subsection{Private and public functions}

Both member functions and global functions may be declared private or public by
prefixing the function definition with the keyword \code{private} or
\code{public}. Private functions are only accessible from the file they're
declared in. Public functions are accessible from anywhere, including other
modules. Functions not marked private or public are
\nonterminal{module-private}, i.e. only accessible within the module they're
declared in.

\subsection{Function specifiers}
\label{sec:function-specifiers}

\begin{grammar}
\rule{function-specifier} \code{inline}
\end{grammar}

\subsubsection{\code{inline}}

A function defined with the \code{inline} keyword is an \textit{inline
function}. Inline functions are guaranteed to be inlined when compiling in debug
mode (without optimizations). When compiling with optimizations, the compiler
may choose to not inline an inline function.

\section{Structs}

Structs are defined as follows:

\begin{grammar}
\rule{struct-definition} \code{struct} \nonterminal{struct-name} \code{\{} \nonterminal{member-list} \code{\}}
\end{grammar}

\nonterminal{struct-name} becomes the name of the struct.
\nonterminal{member-list} is a sequence of
\nonterminal{member-variable-declarations} and
\nonterminal{member-function-declarations}. Structs can be declared to implement
interfaces by listing the interfaces after a \code{:} following the struct name:

\begin{grammar}
\code{struct} \nonterminal{struct-name} \code{:} \nonterminal{interface-list} \code{\{} \nonterminal{member-list} \code{\}}
\end{grammar}

The \nonterminal{interface-list} is a comma-separated list of one or more
interface names. The compiler will emit an error if the struct doesn't fulfill
all the requirements of a specified interface.

\subsection{Member variables}

Structs can contain member variables. The syntax of a member variable definition
is as follows:

\begin{grammar}
\rule{let-or-var} \code{let} | \code{var}\\
\rule{member-variable-declaration} \nonterminal{let-or-var} \nonterminal{member-variable-name} \code{:} \nonterminal{type} \code{;}
\end{grammar}

\subsection{Generic structs}

Generic structs can be declared as follows:

\begin{grammar}
\code{struct} \nonterminal{struct-name} \code{<} \nonterminal{generic-parameter-list} \code{>} \code{\{} \nonterminal{member-list} \code{\}}
\end{grammar}

where \nonterminal{generic-parameter-list} is a comma separated list of one or
more \nonterminal{generic-parameters}:

\begin{grammar}
\rule{generic-parameter} \nonterminal{generic-type-parameter}\\
\rule{generic-type-parameter} \nonterminal{identifier}
\end{grammar}

The identifier of a \nonterminal{generic-type-parameter} serves as a placeholder
for types used to instantiate the generic struct.
