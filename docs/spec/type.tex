%!TEX root = spec.tex

\chapter{Types}

\section{Integer types}

There are eight built-in integer types: \code{int8}, \code{int16}, \code{int32},
\code{int64}, and their unsigned counterparts \code{uint8}, \code{uint16},
\code{uint32}, \code{uint64}. The standard library also provides \code{int} and
\code{uint} which are type aliases for \code{int32} and \code{uint32},
respectively.

\section{Floating-point types}

There are three built-in floating-point types: \code{float32}, \code{float64},
and \code{float80}. The standard library also provides \code{float} and
\code{double} which are type aliases for \code{float32} and \code{float64},
respectively.

\section{Pointer types}

Pointers are values that point to other values. They can be reassigned to point
to another value (if the pointer type itself is declared as \code{mutable}), but
they must always refer to some value, i.e. they cannot be null by default
(nullable pointers can be created using the optional type, see below). Member
access, member function calls, and subscript operations via pointers are
allowed: they will be forwarded to the pointee value.

\begin{grammar}
\nonterminal{pointer-type} \textrightarrow{} \nonterminal{pointee-type} \code{*}\\
\nonterminal{pointer-type} \textrightarrow{} \nonterminal{pointee-type} \code{mutable} \code{*}\\
\nonterminal{pointer-type} \textrightarrow{} \code{mutable} \nonterminal{pointee-type} \code{*}\\
\nonterminal{pointer-type} \textrightarrow{} \code{mutable} \nonterminal{pointee-type} \code{mutable} \code{*}
\end{grammar}

Prefixing the \nonterminal{pointee-type} with \code{mutable} makes the
\nonterminal{pointee-type} mutable. Prefixing the \code{*} with \code{mutable}
makes the \nonterminal{pointer-type} itself mutable.

\begin{samepage}
Pointer arithmetic is supported in the form of the following operations:

\begin{itemize}
\item \nonterminal{pointer} \code{+} \nonterminal{integer}
\item \nonterminal{pointer} \code{+=} \nonterminal{integer}
\item \nonterminal{pointer} \code{++}
\item \nonterminal{pointer} \code{-} \nonterminal{integer}
\item \nonterminal{pointer} \code{-=} \nonterminal{integer}
\item \nonterminal{pointer} \code{--}
\item \nonterminal{pointer} \code{-} \nonterminal{pointer}
\end{itemize}
\end{samepage}

\section{Array types}

\begin{grammar}
\nonterminal{array-type-with-constant-size} \textrightarrow{} \nonterminal{element-type} \code{[} \nonterminal{size} \code{]}\\
\nonterminal{array-type-with-runtime-size} \textrightarrow{} \nonterminal{element-type} \code{[} \code{]}\\
\nonterminal{array-type-with-unknown-size} \textrightarrow{} \nonterminal{element-type} \code{[} \code{?} \code{]}
\end{grammar}

\nonterminal{array-type-with-constant-size} represents a contiguous block of
\nonterminal{size} elements of type \nonterminal{element-type}.
\nonterminal{array-type-with-runtime-size} is conceptually a pointer-and-size
pair. \nonterminal{array-type-with-unknown-size} can only be used through a
pointer; such pointers are memory-layout-compatible with pointers to
\nonterminal{element-type}, primarily for C interoperability.
\nonterminal{array-type-with-unknown-size} is the only array type for which
index operations are not guaranteed to be bounds-checked.

\section{String type}

The type \code{String} holds sequences of Unicode characters.

\section{Struct types}

Structs are composite data types which can be defined using the \code{struct}
keyword.

\subsection{Member variables}

Structs types contain a set of member variables. The syntax of a member variable
definition is as follows:

\begin{grammar}
\nonterminal{let-or-var} \textrightarrow{} \code{let} | \code{var}\\
\nonterminal{member-variable-declaration} \textrightarrow{} \nonterminal{let-or-var} \nonterminal{member-variable-name} \code{:} \nonterminal{type} \code{;}
\end{grammar}

\section{Interface types}

The \code{interface} keyword declares an interface, i.e. a set of requirements
(member functions and properties). Types that are declared to implement an
interface \code{I} and fulfill \code{I}'s requirements can be used as values for
a variable of type \code{I}. This enables runtime polymorphism. Like structs,
interfaces may be generic.

\section{Optional type}

\begin{quote}
``I call it my billion-dollar mistake. It was the invention of the null
reference in 1965. At that time, I was designing the first comprehensive type
system for references in an object oriented language (ALGOL W). My goal was to
ensure that all use of references should be absolutely safe, with checking
performed automatically by the compiler. But I couldn't resist the temptation to
put in a null reference, simply because it was so easy to implement. This has
led to innumerable errors, vulnerabilities, and system crashes, which have
probably caused a billion dollars of pain and damage in the last forty years.''

--- C. A. R. Hoare
\end{quote}

An object of the optional type \code{T?} (where \code{T} is an arbitrary type)
may contain a value of type \code{T} or the value \code{null}.

\begin{grammar}
\nonterminal{optional-type} \textrightarrow{} \nonterminal{wrapped-type} \code{?}
\end{grammar}

\section{Function types}

Function types are written out as follows:

\begin{grammar}
\nonterminal{function-type} \textrightarrow{} \code{(} \nonterminal{parameter-type-list} \code{)} \code{->} \nonterminal{return-type}
\end{grammar}

where \nonterminal{parameter-type-list} is a comma-separated list of parameter
types.

\section{Tuple types}

\begin{grammar}
\nonterminal{tuple-type} \textrightarrow{} \code{(} \nonterminal{tuple-element-list} \code{)}\\
\nonterminal{tuple-element-list} \textrightarrow{} comma-separated list of one or more \nonterminal{tuple-elements}\\
\nonterminal{tuple-element} \textrightarrow{} \nonterminal{name} \code{:} \nonterminal{type}
\end{grammar}

Tuples behave like structs, but they're defined inline. Tuples are intended as
a lightweight alternative for situations where defining a whole new struct feels
overkill or inappropriate, e.g. returning multiple values from a function.

While struct types are considered the same only if they have the same name,
tuple types are considered the same if their structure is the same, i.e. if they
have the same number of elements in the same order with the same names and
types.

\subsection{Tuple unpacking}

The elements of a tuple value may be unpacked into individual variables as
follows:

\begin{grammar}
\nonterminal{tuple-unpack-statement} \textrightarrow{} \nonterminal{variable-list} \code{=} \nonterminal{tuple-expression} \code{;}
\end{grammar}

\nonterminal{variable-list} is a comma-separated list of one or more variable
names. The variable names must match the element names of the
\nonterminal{tuple-expression}, and be in the same order.

\section{Range types}

The standard library defines the following two generic types to represent
ranges:

\begin{itemize}
\item \code{Range<T>} for ranges with an exclusive upper bound
\item \code{ClosedRange<T>} for ranges with an inclusive upper bound
\end{itemize}
