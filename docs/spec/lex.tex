%!TEX root = spec.tex

\chapter{Lexical structure}

\section{Keywords}

The following keywords are reserved and can't be used as identifiers.

\code{addressof}, \code{break}, \code{case}, \code{catch}, \code{const}, \code{continue}, \code{default}, \code{defer}, \code{deinit},
\code{do}, \code{else}, \code{enum}, \code{extern}, \code{fallthrough}, \code{false}, \code{for}, \code{goto}, \code{if}, \code{import},
\code{in}, \code{init}, \code{inline}, \code{interface}, \code{null}, \code{private}, \code{public}, \code{return}, \code{sizeof},
\code{static}, \code{struct}, \code{switch}, \code{this}, \code{throw}, \code{throws}, \code{true}, \code{try}, \code{typealias},
\code{undefined}, \code{var}, \code{while}, \code{\_}

\section{Operators and delimiters}

Binary arithmetic operators:

\code{+ - * / ** \% \& \&\& | || \string^ << >> += -= *= /= **= \%= \&= \&\&= |= ||= \string^= <<= >>=}

Binary comparison operators:

\code{== != < > <= >=}

Miscellaneous binary operators:

\code{= .. ...}

Unary prefix operators:

\code{+ - * \& ! \~{}}

Unary postfix operators:

\code{++ -- !}

Delimiters:

\code{( [ \{ ) ] \} . , : ; ->}

From the above sets of operators, the following are overloadable by user code:
\code{+} (both unary and binary), \code{-} (both unary and binary), \code{*},
\code{/}, \code{\%}, \code{==}, \code{<}.

\section{Comments}

Delta has two kinds of comments:

\begin{itemize}
    \item Line comments that start with \code{//} and continue until the end of the line.
    \item Block comments that start with \code{/*} and end with \code{*/}. Block comments can be nested.
\end{itemize}

\section{Literals}

\subsection{Integer literal}

\begin{grammar}
    \rule{binary-digit} \code{0} | \code{1}\\
    \rule{octal-digit} \code{0} | \code{1} | \code{2} | \code{3} | \code{4} | \code{5} | \code{6} | \code{7}\\
    \rule{decimal-digit} \code{0} | \code{1} | \code{2} | \code{3} | \code{4} | \code{5} | \code{6} | \code{7} | \code{8} | \code{9}\\
    \rule{nonzero-decimal-digit} \code{1} | \code{2} | \code{3} | \code{4} | \code{5} | \code{6} | \code{7} | \code{8} | \code{9}\\
    \rule{lowercase-hex-digit} \code{0} | \code{1} | \code{2} | \code{3} | \code{4} | \code{5} | \code{6} | \code{7} | \code{8} | \code{9} | \code{a} | \code{b} | \code{c} | \code{d} | \code{e} | \code{f}\\
    \rule{uppercase-hex-digit} \code{0} | \code{1} | \code{2} | \code{3} | \code{4} | \code{5} | \code{6} | \code{7} | \code{8} | \code{9} | \code{A} | \code{B} | \code{C} | \code{D} | \code{E} | \code{F}

    \rule{binary-integer-literal} \code{0b}\nonterminal{binary-digit}+\\
    \rule{octal-integer-literal} \code{0o}\nonterminal{octal-digit}+\\
    \rule{decimal-integer-literal} \nonterminal{nonzero-decimal-digit} \nonterminal{decimal-digit}* | \code{0}\\
    \rule{hex-integer-literal} \code{0x}(\nonterminal{lowercase-hex-digit}+ | \nonterminal{uppercase-hex-digit}+)

    \rule{integer-literal} \nonterminal{binary-integer-literal}\\
    \rule{integer-literal} \nonterminal{octal-integer-literal}\\
    \rule{integer-literal} \nonterminal{decimal-integer-literal}\\
    \rule{integer-literal} \nonterminal{hex-integer-literal}
\end{grammar}

\subsection{Floating-point literal}

Floating-point literals have the following form:

\begin{grammar}
    \rule{floating-point-literal} \nonterminal{nonzero-decimal-digit} \nonterminal{decimal-digit}*\code{.}\nonterminal{decimal-digit}+\\
    \rule{floating-point-literal} \code{0}\code{.}\nonterminal{decimal-digit}+
\end{grammar}

\subsection{Boolean literal}

\begin{grammar}
    \rule{boolean-literal} \code{true} | \code{false}
\end{grammar}

\subsection{Null literal}

\begin{grammar}
    \rule{null-literal} \code{null}
\end{grammar}

\subsection{String literal}

\begin{grammar}
    \rule{string-literal} \code{"}(\nonterminal{character} | \nonterminal{interpolated-expression})*\code{"}\\
    \rule{interpolated-expression} \code{\$\{} \nonterminal{expression} \code{\}}
\end{grammar}

\subsection{Array literal}

\begin{grammar}
\rule{array-literal} \code{[} \nonterminal{elements} \code{]}
\end{grammar}

where \nonterminal{elements} is a comma-separated list of zero or more expressions of the
same type.

\subsection{Tuple literal}

\begin{grammar}
\rule{tuple-literal} \code{(} \nonterminal{elements} \code{)}
\end{grammar}

where \nonterminal{elements} is a comma-separated list of zero or more
\nonterminal{tuple-literal-elements}:

\begin{grammar}
\rule{tuple-literal-element} \nonterminal{identifier} \code{:} \nonterminal{expression}\\
\rule{tuple-literal-element} \nonterminal{identifier}
\end{grammar}

The second form is a shorthand for \nonterminal{tuple-literal-elements} of the
form `\nonterminal{identifier} \code{:} \nonterminal{identifier}'.

\section{Identifiers}

\begin{grammar}
\rule{identifier-first-character} upper- or lowercase letter \code{A} through \code{Z}\\
\rule{identifier-first-character} \code{\_}

\rule{identifier-character} \nonterminal{identifier-first-character}\\
\rule{identifier-character} \code{0} | \code{1} | \code{2} | \code{3} | \code{4} | \code{5} | \code{6} | \code{7} | \code{8} | \code{9}

\rule{identifier} \nonterminal{identifier-first-character} \nonterminal{identifier-character}*
\end{grammar}
