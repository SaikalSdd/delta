%!TEX root = spec.tex

\chapter{Lexical structure}

\section{Keywords}

The following keywords are reserved and can't be used as identifiers.

\begin{multicols}{3}
\begin{itemize}[label={}]
\item \code{addressof}
\item \code{break}
\item \code{case}
\item \code{catch}
\item \code{const}
\item \code{continue}
\item \code{default}
\item \code{defer}
\item \code{deinit}
\item \code{do}
\item \code{else}
\item \code{enum}
\item \code{extern}
\item \code{fallthrough}
\item \code{false}
\item \code{for}
\item \code{func}
\item \code{goto}
\item \code{if}
\item \code{import}
\item \code{in}
\item \code{init}
\item \code{interface}
\item \code{let}
\item \code{mutable}
\item \code{mutating}
\item \code{null}
\item \code{private}
\item \code{public}
\item \code{return}
\item \code{sizeof}
\item \code{static}
\item \code{struct}
\item \code{switch}
\item \code{this}
\item \code{throw}
\item \code{throws}
\item \code{true}
\item \code{try}
\item \code{typealias}
\item \code{undefined}
\item \code{var}
\item \code{while}
\item \code{_}
\end{itemize}
\end{multicols}

\section{Operators and delimiters}

Binary arithmetic operators:

\begin{multicols}{4}
\begin{itemize}[label={}]
\item \code{+}
\item \code{-}
\item \code{*}
\item \code{/}
\item \code{**}
\item \code{\%}
\item \code{\&}
\item \code{\&\&}
\item \code{|}
\item \code{||}
\item \code{\char`\^}
\item \code{<<}
\item \code{>>}
\item \code{+=}
\item \code{-=}
\item \code{*=}
\item \code{/=}
\item \code{**=}
\item \code{\%=}
\item \code{\&=}
\item \code{\&\&=}
\item \code{|=}
\item \code{||=}
\item \code{\char`\^=}
\item \code{<<=}
\item \code{>>=}
\end{itemize}
\end{multicols}

Binary comparison operators:

\begin{multicols}{4}
\begin{itemize}[label={}]
\item \code{==}
\item \code{!=}
\item \code{<}
\item \code{>}
\item \code{<=}
\item \code{>=}
\end{itemize}
\end{multicols}

Miscellaneous binary operators:

\begin{multicols}{4}
\begin{itemize}[label={}]
\item \code{=}
\item \code{..}
\item \code{...}
\end{itemize}
\end{multicols}

Unary prefix operators:

\begin{multicols}{4}
\begin{itemize}[label={}]
\item \code{+}
\item \code{-}
\item \code{*}
\item \code{\&}
\item \code{!}
\item \code{\~}
\end{itemize}
\end{multicols}

Unary postfix operators:

\begin{multicols}{4}
\begin{itemize}[label={}]
\item \code{++}
\item \code{--}
\item \code{!}
\end{itemize}
\end{multicols}

Delimiters:

\begin{multicols}{4}
\begin{itemize}[label={}]
\item \code{(}
\item \code{[}
\item \code{\{}
\item \code{)}
\item \code{]}
\item \code{\}}
\item \code{.}
\item \code{,}
\item \code{:}
\item \code{;}
\item \code{->}
\end{itemize}
\end{multicols}

From the above sets of operators, the following are overloadable by user code:
\code{+} (both unary and binary), \code{-} (both unary and binary), \code{*},
\code{/}, \code{\%}, \code{==}, \code{<}.

\section{Comments}

Delta has two kinds of comments:

\begin{itemize}
\item Line comments that start with \code{//} and continue until the end of the line.
\item Block comments that start with \code{/*} and end with \code{*/}. Block comments can be nested.
\end{itemize}

\section{Literals}

\subsection{Integer literal}

\begin{grammar}
\nonterminal{binary-digit} \textrightarrow{} \code{0} | \code{1}\\
\nonterminal{octal-digit} \textrightarrow{} \code{0} | \code{1} | \code{2} | \code{3} | \code{4} | \code{5} | \code{6} | \code{7}\\
\nonterminal{decimal-digit} \textrightarrow{} \code{0} | \code{1} | \code{2} | \code{3} | \code{4} | \code{5} | \code{6} | \code{7} | \code{8} | \code{9}\\
\nonterminal{nonzero-decimal-digit} \textrightarrow{} \code{1} | \code{2} | \code{3} | \code{4} | \code{5} | \code{6} | \code{7} | \code{8} | \code{9}\\
\nonterminal{lowercase-hex-digit} \textrightarrow{} \code{0} | \code{1} | \code{2} | \code{3} | \code{4} | \code{5} | \code{6} | \code{7} | \code{8} | \code{9} | \code{a} | \code{b} | \code{c} | \code{d} | \code{e} | \code{f}\\
\nonterminal{uppercase-hex-digit} \textrightarrow{} \code{0} | \code{1} | \code{2} | \code{3} | \code{4} | \code{5} | \code{6} | \code{7} | \code{8} | \code{9} | \code{A} | \code{B} | \code{C} | \code{D} | \code{E} | \code{F}

\nonterminal{binary-integer-literal} \textrightarrow{} \code{0b}\nonterminal{binary-digit}+\\
\nonterminal{octal-integer-literal} \textrightarrow{} \code{0o}\nonterminal{octal-digit}+\\
\nonterminal{decimal-integer-literal} \textrightarrow{} \nonterminal{nonzero-decimal-digit} \nonterminal{decimal-digit}* | \code{0}\\
\nonterminal{hex-integer-literal} \textrightarrow{} \code{0x}(\nonterminal{lowercase-hex-digit}+ | \nonterminal{uppercase-hex-digit}+)

\nonterminal{integer-literal} \textrightarrow{} \nonterminal{binary-integer-literal}\\
\nonterminal{integer-literal} \textrightarrow{} \nonterminal{octal-integer-literal}\\
\nonterminal{integer-literal} \textrightarrow{} \nonterminal{decimal-integer-literal}\\
\nonterminal{integer-literal} \textrightarrow{} \nonterminal{hex-integer-literal}
\end{grammar}

\subsection{Floating-point literal}

Floating-point literals have the following form:

\begin{grammar}
\nonterminal{floating-point-literal} \textrightarrow{} \nonterminal{nonzero-decimal-digit} \nonterminal{decimal-digit}*\code{.}\nonterminal{decimal-digit}+\\
\nonterminal{floating-point-literal} \textrightarrow{} \code{0}\code{.}\nonterminal{decimal-digit}+
\end{grammar}

\subsection{Boolean literal}

\begin{grammar}
\nonterminal{boolean-literal} \textrightarrow{} \code{true} | \code{false}
\end{grammar}

\subsection{Null literal}

\begin{grammar}
\nonterminal{null-literal} \textrightarrow{} \code{null}
\end{grammar}

\subsection{String literal}

\begin{grammar}
\nonterminal{string-literal} \textrightarrow{} \code{"}(\nonterminal{character} | \nonterminal{interpolated-expression})*\code{"}\\
\nonterminal{interpolated-expression} \textrightarrow{} \code{\$\{} \nonterminal{expression} \code{\}}
\end{grammar}

\subsection{Array literal}

\begin{grammar}
\nonterminal{array-literal} \textrightarrow{} \code{[} \nonterminal{elements} \code{]}
\end{grammar}

where \nonterminal{elements} is a comma-separated list of zero or more expressions of the
same type.

\subsection{Tuple literal}

\begin{grammar}
\nonterminal{tuple-literal} \textrightarrow{} \code{(} \nonterminal{elements} \code{)}
\end{grammar}

where \nonterminal{elements} is a comma-separated list of zero or more
\nonterminal{tuple-literal-elements}:

\begin{grammar}
\nonterminal{tuple-literal-element} \textrightarrow{} \nonterminal{identifier} \code{:} \nonterminal{expression}\\
\nonterminal{tuple-literal-element} \textrightarrow{} \nonterminal{identifier}
\end{grammar}

The second form is a shorthand for \nonterminal{tuple-literal-elements} of the
form `\nonterminal{identifier} \code{:} \nonterminal{identifier}'.

\section{Identifiers}

\begin{grammar}
\nonterminal{identifier-first-character} \textrightarrow{} upper- or lowercase letter \code{A} through \code{Z}\\
\nonterminal{identifier-first-character} \textrightarrow{} \code{_}

\nonterminal{identifier-character} \textrightarrow{} \nonterminal{identifier-first-character}\\
\nonterminal{identifier-character} \textrightarrow{} \code{0} | \code{1} | \code{2} | \code{3} | \code{4} | \code{5} | \code{6} | \code{7} | \code{8} | \code{9}

\nonterminal{identifier} \textrightarrow{} \nonterminal{identifier-first-character} \nonterminal{identifier-character}*
\end{grammar}
