%!TEX root = spec.tex

\chapter{Statements}

\section{Assignment statement}

\begin{grammar}
\rule{assignment-statement} \nonterminal{lvalue-expression} \code{=} \nonterminal{expression} \code{;}\\
\rule{assignment-statement} \code{\_} \code{=} \nonterminal{expression} \code{;}
\end{grammar}

Assignments in Delta don't return any value. This applies to compound
assignments as well, including \code{++} and \code{--} (see below). Furthermore,
this obsoletes the two different forms of \code{++} and \code{--}, so only the
postfix versions are valid as syntactic sugar for \code{+= 1} and \code{-= 1},
respectively.

The assignment to \code{\_}, called the \nonterminal{discarding-assignment}, can
be used to ignore the result of the right-hand side expression, suppressing any
compilation errors or warnings that would otherwise be emitted.

\section{Increment and decrement statements}

\begin{grammar}
\rule{increment-statement} \nonterminal{lvalue-expression} \code{++} \code{;}\\
\rule{decrement-statement} \nonterminal{lvalue-expression} \code{--} \code{;}
\end{grammar}

\section{Block}

\begin{grammar}
\rule{block} \code{\{} \nonterminal{statement}* \code{\}}
\end{grammar}

\section{\code{if} statement}

\begin{grammar}
\rule{if-statement} \code{if} \code{(} \nonterminal{expression} \code{)} \nonterminal{block}\\
\rule{if-statement} \code{if} \code{(} \nonterminal{expression} \code{)} \nonterminal{block} \code{else} \nonterminal{block}\\
\rule{if-statement} \code{if} \code{(} \nonterminal{expression} \code{)} \nonterminal{block} \code{else} \nonterminal{if-statement}
\end{grammar}

\section{\code{return} statement}

\begin{grammar}
\rule{return-statement} \code{return} \nonterminal{expression}\optional{} \code{;}
\end{grammar}

\section{\code{for} statement}

The \code{for} statement loops over a range. The syntax is as follows:

\begin{grammar}
\rule{for-statement} \code{for} \code{(} \code{var} \nonterminal{identifier} \code{in} \nonterminal{range-expression} \code{)} \nonterminal{block}
\end{grammar}

\section{\code{while} statement}

The \code{while} statement loops until a condition evaluates to \code{false}.
The syntax is as follows:

\begin{grammar}
\rule{while-statement} \code{while} \code{(} \nonterminal{condition} \code{)} \nonterminal{block}
\end{grammar}

\section{\code{switch} statement}

\begin{grammar}
\rule{switch-statement} \code{switch} \code{(} \nonterminal{expression} \code{)} \code{\{} \nonterminal{case}+ \code{\}}\\
\rule{case} \code{case} \nonterminal{expression} \code{:} \nonterminal{statement}+\\
\rule{case} \code{default} \code{:} \nonterminal{statement}+
\end{grammar}

In addition to integer types, the \code{switch} statement can be used to match
strings.

The cases in a \code{switch} statement don't fall through by default. The
fall-through behavior can be enabled for a individual cases with the
\code{fallthrough} keyword.

\code{switch} statements must be exhaustive if \nonterminal{expression} is of an
enum type. This is enforced by the compiler.

\section{\code{defer} statement}

The \code{defer} statement has the following syntax:

\begin{grammar}
\rule{defer-statement} \code{defer} \nonterminal{block}
\end{grammar}

The \nonterminal{block} will be executed when leaving the scope where the
\nonterminal{defer-statement} is located. Multiple deferred blocks are executed
in the reverse of the order they were declared in. Return statements are
disallowed inside the defer block.
